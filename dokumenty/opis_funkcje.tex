\documentclass[2pt]{article}

\usepackage[utf8x]{inputenc}
\usepackage{ucs}
\usepackage[MeX]{polski}
\usepackage{fancyhdr}
\usepackage{amsmath}
\usepackage{amsfonts}
\usepackage{amssymb}
\usepackage{graphicx}
\usepackage{indentfirst}
\usepackage{caption}
\usepackage{subcaption}
\usepackage{listings}
\usepackage{lscape}
\newcommand\V[1]{\textup{#1}}

\begin{document}
\title{Sprawozdanie}
\author{Adam Prochownik}

\date{26 marca 2015}

\section{dane firmy}
Dodaj firmę - dodaję firmę, jako argumenty przyjmuję Nazwę i opis \\

Usuń firmę - usuwa firmę, wszystkie jej zadania itp. argumenty: id firmy. \\

Edytuj firmę - zmienia nazwę firmy i opis, argumenty: id firmy,  nazwa, opis. \\

\section{zadania}
Dodaj zadanie - dodaje zadanie, uzupełnia datę przyjęcia, uzupełnia operacje przypisane do zadania. Argumenty: operacje, id firmy, id maszyn. \\

Usuń zadania - usuwa pojedyncze zadanie, razem z wszystkimi przypisanego do niego operacjami, oraz znalezionymi permutacjami oraz przypisanymi maszynami. Argumenty: pojedyncze zadanie. \\

\section{maszyny}
Dodaj maszynę - dodaję nową maszynę do systemu. Argumenty: opis maszyny. \\

Usuń maszynę - usuwa maszynę, razem ze wszystkimi procesami i zadaniami, które ją wykorzystują. Argumenty: id maszyny.\\

Sprawdź\_czy\_używana - sprawdza, czy dana maszyna jest używana w jakimś zadaniu/procesie. Argumenty: id maszyna. \\

\section{permutacja\_operacje}
Policz permutację - oblicza permutację i uzupełnia datę obliczenia w tabeli zadania. Argumenty: id zadania. \\

\end{document}